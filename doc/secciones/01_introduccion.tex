\chapter{Introducción}

El objetivo de este proyecto es la creación de un software capaz de transformar de forma autonoma un modelo 3D detallado de una escultura cualquiera en una escultura que inmita el estilo del escultor Pablo Emilio Gargallo Catalán.
Este estilo se caracteriza principalmente por jugar con las concavidades y las convexidades de las esculturas para obtener efectos no obtenibles por medio de tecnicas tradicionales. Nuestro punto focal de interes va a ser precisamente
lograr transformar de forma automática las convexidades de los modelos 3D en concavidades. Para permitirnos hacer esto vamos a utilizar varios recursos de los graficos por ordenador.

El termino graficos por ordenador se refiere a un campo multidisciplinar donde se agrupa cualquier disciplina que contribuye a la creación y visualización de representaciones pictoricas enteremante en un ordenador.
Esto engloba una variedad de campos.Los relevantes para nuestro proyecto 
\begin{list}{.}{}
    \item El modelado, que es la rama que lidia con la representación matematica de la especificación de la forma y la apariencia de una representación pictorica concreta de forma que sea interpetable para un ordenador
    \item El Renderizado, que lidia con la creación de las representaciones 2D que visualizamos en la pantalla del espacio tridimensional con el que trabajamos
\end{list} \cite{marschner_fundamentals_2018}


Este proyecto es software libre, y está liberado con la licencia \cite{gplv3}.